\documentclass[a4paper,12pt]{article}

\usepackage[utf8]{inputenc}
\usepackage[T1]{fontenc}
\usepackage[english]{babel}
\usepackage{geometry}
\usepackage{setspace}
\usepackage{hyperref}
\usepackage{titlesec}
\usepackage{enumitem}
\usepackage{parskip}
\usepackage{titling}

\geometry{margin=2.5cm}
\setlength{\droptitle}{-4em}
\setstretch{1.15}

\title{\textbf{Visual Data Science - Discover Report}\\[0.2em]\smaller{}Global Electricity Production and Composition Overview}
\author{Philipp Wimmer, 12536839}
\date{\today}

\begin{document}

\maketitle

\section{Topic Description}

The focus of my project is to visualize and analyze global electricity production and its composition over time.
The main objectives are to explore how different countries generate electricity, how the share of renewable and non-renewable sources has evolved
and how these trends relate to environmental impact, particularly CO\textsubscript{2} emissions.
Special attention will be given to Austria, Germany, and the top 20 countries worldwide by total electricity generation.
The analysis will examine the transition towards renewable energy, identifying leading nations in renewable adoption
and observing significant developments such as the rapid rise of solar and wind power in recent decades.
My project aims to highlight trends, shifts, and milestones in the global energy landscape, linking them to broader sustainability goals.

Visualizations will include:
\begin{itemize}[noitemsep]
    \item Line charts showing the growth of renewables over time
    \item Stacked area charts comparing renewable vs. non-renewable sources
    \item Bar charts comparing countries or regions
    \item Geographical maps to visualize spatial differences in electricity generation and emissions
\end{itemize}

Through these visualizations, my project wants to provide clear insights into how the global electricity mix is changing and how this affects CO\textsubscript{2} emissions and climate objectives.

\section{Dataset Description}

The dataset used for my project originates from the \textbf{Energy Institute Statistical Review of World Energy}
(\href{https://www.energyinst.org/statistical-review/home}{https://www.energyinst.org/statistical-review/home}).
It provides comprehensive global data on energy production, consumption, reserves, and emissions across all major energy sources.
The dataset is available both as an Excel file with multiple sheets and as a consolidated CSV file,
which merges all relevant information into a single structured format with country-year observations.
For this analysis, the focus is on columns related to electricity generation and CO\textsubscript{2} emissions,
which allow for evaluating energy trends, fuel mix, and environmental impact at a national level.

\subsection*{Columns of Interest}

\textbf{Electricity Generation}
\begin{itemize}[noitemsep]
    \item \texttt{elect\_twh} / \texttt{electbyfuel\_total}: Total Electricity Generation
    \item \texttt{electbyfuel\_coal}: Electricity Generation from Coal
    \item \texttt{electbyfuel\_gas}: Electricity Generation from Natural Gas
    \item \texttt{electbyfuel\_oil}: Electricity Generation from Oil
    \item \texttt{electbyfuel\_nuclear}: Electricity Generation from Nuclear
    \item \texttt{electbyfuel\_hydro} / \texttt{hydro\_twh}: Electricity Generation from Hydro
    \item \texttt{electbyfuel\_ren\_power}: Electricity Generation from Renewables (excluding hydro)
    \begin{itemize}
        \item \texttt{wind\_twh}: Wind Generation
        \item \texttt{solar\_twh}: Solar Generation
    \end{itemize}
    \item \texttt{electbyfuel\_other}: Electricity Generation from Other Sources (e.g., biomass, geothermal)
    \begin{itemize}
        \item \texttt{biogeo\_twh}: Geothermal, Biomass, and Other Renewable Generation
    \end{itemize}
\end{itemize}

\textbf{CO\textsubscript{2}}
\begin{itemize}[noitemsep]
    \item \texttt{co2\_combust\_mtco2}: CO\textsubscript{2} Emissions from Energy Combustion
    \item \texttt{co2\_combust\_pc}: CO\textsubscript{2} Emissions per Capita
\end{itemize}

This dataset provides a robust foundation for analyzing electricity generation patterns, comparing renewable vs.\ non-renewable sources, and assessing the relationship between electricity production and CO\textsubscript{2} emissions on both national and global scales.

\end{document}
